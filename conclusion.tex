In this paper, we have presented LM, a new linear logic programming language designed with concurrency in mind. LM is a bottom-up logic programming language
that can naturally model state due to its foundations on linear logic. While it is targeted to solve graph-based algorithms, it can also model other
types of programs as long as the problem can be mapped to a graph.

We presented several programs written in LM that show the viability of a linear logic based programming language to solve interesting problems. Our experimental
results show that LM programs can run efficiently and scale well on multicore architectures, freeing the programmer from known pitfalls in parallel programming.

We also gave an overview of the formal system behind LM, namely, the fragment of linear logic used in the language, along with the high level and low level dynamic semantics.
While the former is closely tied to linear logic, the latter is closer to a real implementation. The low level dynamic semantics can be used as a blue print for someone that
intends to implement our language, since it also informed the design of our virtual machine.