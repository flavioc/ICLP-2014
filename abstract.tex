Due to the popularity of social networks on the internet, there has
been an increased interested in executing graph based algorithms
concurrently. Systems such as Pregel or GraphLab are specialized
frameworks for programming and executing algorithms over graph
structures. Although these frameworks make it significantly easier to
write those kinds of programs, they tend to have a steep learning
curve because they use low level languages such as C++. We have
designed a new logic programming language called LM (Linear Meld) for
programming graph based algorithms in a declarative fashion. Logic
based languages raise the programming abstractions because they use
logical clauses to drive computation, which makes them be both easier
to write and to reason about. Since our language is based on linear
logic, a powerful logical system where logical facts can be consumed,
our programs tend to be more expressive than in other logic
programming languages. Programs are mapped to an underlying graph
structure where computation is performed at the node level and
communication takes place when nodes send facts to each other.  LM
programs are thus naturally concurrent because facts are partitioned
by nodes of a graph data structure. In this paper we want to
present the syntax and semantics of our language, explain how programs
can be written and what novel ideas LM brings to the table.
