
\newcommand{\mz}{\m{match} \;}
\newcommand{\tab}[0]{\;\;\;\;}
\newcommand{\dz}{\m{derive} \;}
\newcommand{\comp}[0]{\m{comp} \;}
\newcommand{\az}{\m{apply} \;}
\newcommand{\doz}{\m{run} \;}
\newcommand{\seqnocut}[3]{#1 ; #2 \Rightarrow #3}
\newcommand{\defeq}{\buildrel\triangle\over =}
\newcommand{\compr}[1]{\m{def} \; #1}

\newcommand{\mo}{\m{match}_1 \;}
\newcommand{\cont}{\m{cont} \;}
\newcommand{\contc}{\m{contc} \;}
\newcommand{\done}{\m{derive}_1 \;}
\newcommand{\doo}{\m{run}_1 \;}
\newcommand{\mc}[0]{\m{match}_c \; }
\newcommand{\dall}[0]{\m{fix} \; }
\newcommand{\strans}[0]{\m{strans} \;}
\newcommand{\dc}{\m{derive}_c \;}
\newcommand{\ao}{\m{apply}_1 \;}

In this section, we present the sequent calculus of a fragment of intuitionistic linear logic~\cite{Girard95logic:its} used by LM and also the dynamic semantics of LM. For the latter, we formalized both an higher level dynamic semantics based on the linear logic fragment and a more concrete low level dynamic semantics that resembles the virtual machine implementation. We also prove that the low level dynamic semantics are sound in respect to the high level semantics. 

\subsection{Intuitionistic Linear Logic}

The fragment of linear logic used by LM is presented in Appendix~\ref{linear_logic}.
We use a standard set of connectives except the $\m{def} \; A$ connective, which is used to logically justify comprehensions and aggregates.
The sequent calculus has the form $\Psi; \Gamma; \Delta \rightarrow C$, where $\Psi$ is the typed term context used in the quantifiers, $\Gamma$ is the multi-set
of persistent terms, $\Delta$ is the multi-set of linear terms and $C$ is the term we want to prove.
Table~\ref{table:linear} relates the linear logic directives with the LM syntax.

For definitions, we took inspiration from Baelde's work on least and greatest fixed points in linear logic~\cite{Baelde:2012:LGF:2071368.2071370}. Baelde's system goes beyond
simple recursive definitions and allows proofs using induction and co-induction in linear logic. Fortunately, the proof system satisfies cut-elimination which
makes it consistent.

In a comprehension, we want to apply an implication as many matches as the database allows. One way to formally describe comprehensions would be to use persistent
rules that would be used a few times and then would be forgotten. A more reasonable approach is to use
definitions. Given a comprehension $C = \{ \; \widehat{x}; \; A; \; B \; \}$ with a body $A$ and a head $B$, then we can build the following recursive definition:

\[
\compr{C} \defeq 1 \with ((A \lolli B) \otimes \compr{C})
\]

We can unfold $\compr{C}$ to either stop (by selecting $1$) or get a new linear implication $A \lolli B$
to apply the comprehension once. Because we also get another $\compr{C}$ by selecting the right hand side,
the comprehension can be applied again, recursively.

Aggregates work identically, but they need an extra argument to accumulate the aggregated value. If a sum aggregate $C$ has the form $[\;\m{sum} \Rightarrow y; \; \widehat{x}; \; A; \; B_1; \; B_2 \;]$, then the definition will be as follows:

\[
\compr{C} \; V \defeq (\lambda v. B_2)V \with (\forall x. ((Ax \lolli B_1) \otimes \compr{C} \; (x + V)))
\]

The aggregate is initiated as $\compr{C} \; 0$.

\footnotesize
\begin{center}
\begin{table}\resizebox{14cm}{!}{
    \begin{tabular}{ c l c l}
    \hline
    Connective                   & Description                                        & LM Place   & LM Example                           \\ \hline \hline
    $\emph{fact}(\hat{x})$       & Linear facts.                                      & Body/Head  & \texttt{path(A, P)}                  \\ 
    $\bang \emph{fact}(\hat{x})$ & Persistent facts.                                  & Body/Head  & \texttt{$\bang$edge(X, Y, W)}        \\ 
    $1$                          & Represents rules with an empty head.               & Head       & \texttt{1}                           \\ 
    $A \otimes B$                & Connect two expressions.                           & Body/Head  & \texttt{p(A), e(A, B)}               \\ 
    $\forall x. A$               & For variables defined in the rule.                 & Rule       & \texttt{p(A) $\lolli$ r(A)}          \\ 
    $\exists x. A$               & Instantiates new node variables.                   & Head       & \texttt{exists A.(p(A, P))}          \\ 
    $A \lolli B$                 & $\lolli$ means "linearly implies".                 & Rule       & \texttt{p(A, B) $\lolli$ r(A, B)}    \\ 
                                 & $A$ is the body and $B$ is the head.               &            &                                      \\ 
    $\m{def} A. B$               & Extension called definitions. Used for             & Head       & \texttt{\{B | !e(A, B) | v(B)\}}     \\ 
                                 & comprehensions and aggregates.                     &            &                                      \\ \hline
    \end{tabular}}
\caption{Connectives from Linear Logic used in LM.}
\label{table:linear}
\end{table}
\end{center}
\normalsize

\subsection{Dynamic Semantics}

The high level dynamic semantics (HLD) formalize the mechanism of matching rules and deriving new facts and is closely related to the fragment presented above.
From the sequent calculus, we consider $\Gamma$ and $\Delta$ the database of persistent and linear facts, respectively.
We consider the rules of the program as persistent linear implications of the form $\bang (A \lolli B)$ that we put in a separate context $\Phi$.
We ignore the right hand side $C$ of the calculus and use inversion on the $\Delta$ and $\Gamma$ contexts so that we only have atomic terms (facts). To apply rules
we use chaining by focusing on the derivation rules of $\Phi$.

The HLD semantics are shown in \ref{hlsemantics} and are composed of four main judgments: (i) $\m{run}$ picks some rule and tries to apply it;
(ii) $\m{apply}$ picks a subset of linear facts from $\Delta$ and tries to match the body of the rule and then derives the head; (iii) $\m{match}$
verifies that a set of linear facts matches the body; and (iv) $\m{derive}$ instantiates the head of the rule and adds facts to $\Delta_1$ and $\Gamma_1$
(the contexts for new persistent and linear facts).

We note that the HLD semantics do not take distribution into account, since we assume that the database is global. We also do not model variable bindings
since it complicates the formal system. The low level dynamic semantics (LLD) shown in \ref{low_level_semantics} improve upon HLD by adding rule priorities, by removing the non-determinism
when matching the body of rules by modeling all the steps and by applying comprehensions or aggregates as many times as the database allows.
LLD is closely related to our virtual machine, thus it is a good starting point to understand our implementation.

In LLD we try all the rules in order. For each rule, we use a \emph{continuation stack} to store the \emph{continuation frames} created by each fact template $p$ present
in the body of the rule. Each frame considers all the facts relevant to the template given the current variable bindings ($\mo$rules), that may or not fail during the
remaining matching process. It contains enough state to resume the matching process at the time of its creation, including which facts of the predicate $p$
we can still try.

Whenever we fail to find a fact to match some part of the body, we fail and force the matching process to backtrack in order to try other alternatives (through $\cont$rules).
If the continuation stack becomes empty, we backtrack again to try the next rule (rule $\cont next \; rule$). On the other hand, if the body of a rule is fully matched (rule $\mo end$),
we end up determining which linear facts from the database are consumed. In HDL this set of facts is \emph{guessed}, but LLD computes them deterministically.

\subsubsection{Comprehensions}

The derivation process in LLD is similar to the one used in HDL, except for the case of comprehensions or aggregates. If we find such construct during
derivation ($\done comp$), we need to create a new continuation stack and start matching the body of the construct exactly as we did with the body of the rule.
When we match the body of the comprehension ($\mc$judgment), we fully derive the head ($\dc$judgment) and then we reuse the continuation stack to find which other combinations of the
database facts can be consumed ($\dc end$). By definition, the continuation stack contains enough information to iterate through all the possible combinations
of the database.

However, in order to reuse the stack, we need to \emph{fix} it by removing all the frames pushed after the first continuation frame
of a linear fact. If we tried to use those frames, we would assumed that the linear facts used by the other frames were still in the database, but that
is not true because they have been consumed during the first application of the comprehension.
For example, if the body is $\bang\mathtt{a(X), b(X), c(X)}$ and the continuation stack has three frames (one per fact), we cannot backtrack to the frame of $\mathtt{c(X)}$
since at that point the matching process was assuming that the previous \texttt{b(X)} linear fact was still available.
Moreover, we also remove the consumed linear facts
from the frames of \texttt{b(X)} and $\bang$\texttt{a(X)} in order to make the stack fully consistent with the new database.
This is performed by rules using the $\strans$and $\dall$judgments.

We finally stop applying the comprehension when the continuation stack is empty ($\contc end$). 
Aggregates use the same mechanism as comprehensions, however we also need to keep track of the accumulated value.

\subsubsection{Soundness}

The soundness theorem proves that if a rule was successfully derived in the LLD semantics then it can also be derived in the HLD semantics. The completeness theorem cannot
be proven since LLD lacks the non-determinism inherent in HLD.

The first main lemma of the soundness proof proves that if we can match the body
of a rule at the low level then we can also match the rule in the high level system using the same database.

\begin{lemma}[Body Match]
Given a match $\mo \Gamma; \Delta_1, \Delta_2; \cdot; A; B; \cdot; R \rightarrow \Xi'; \Delta'; \Gamma'$ that is related to $A$, $\Delta_1, \Delta_2$ and $\Gamma$, we get either:

\begin{enumerate}
   \item $\cont \cdot; B; R; \Gamma; \Xi'; \Delta'; \Gamma'$;
   \item $\mz \Delta_2 \rightarrow A$ and $\mo \Gamma; \Delta_1; \Delta_2; \cdot; B; C'; R \rightarrow \Xi'; \Delta'; \Gamma'$ (related)
\end{enumerate}
\end{lemma}
\begin{proof}Use the main matching soundness theorem.
\end{proof}

When we say that a match is related to a term $A$ and a database $\Delta_1, \Delta_2, \Gamma$ we mean that
the matching judgment is related to the body $A$ of a rule and the initial database is $\Delta_1, \Delta_2, \Gamma$, where $\Gamma$ are
the persistent facts, $\Delta_2$ the linear facts consumed to match the body and $\Delta_1$ the remaining linear facts.
Moreover, the continuation stack is related to $A$ and to the database, therefore it is consistent in relation to them.

The body match lemma tells us that if we start a match of a body $A$ we will either fail (1) and need to try another rule in $R$ or we succeed by building the high level matching judgment $\mz \Delta_2 \rightarrow A$ (the soundness result) and reaching the end of the matching process $\mo \Gamma; \Delta_1; \Delta_2; \cdot; B; C'; R \rightarrow \Xi'; \Delta'; \Gamma'$.

This lemma uses a more complicated theorem that is recursively defined through judgments $\m{match}_1$ and $\m{cont}$ that use mutual induction on the
size of the continuation stack, the size of the remaining terms to match and also the size of alternatives at each continuation frame. We always ensure that the
continuation stack stays consistent at all times, when backtracking or moving forward. Theorem is presented in Appendix~\ref{main_soundness_theorem}.
 
The second stepping stone in the soundness proof is the derivation lemma. After we successfully match the
body of a rule, we need to prove that the derivation process (through judgments $\m{derive}_1$) is also
sound. This lemma is as follows:

\begin{lemma}[Derivation]
   If $\do \Gamma; \Delta; \Xi; \Gamma_1; \Delta_1; \Omega \rightarrow \Xi'; \Delta'; \Gamma'$ then $\dz \Gamma; \Delta; \Gamma_1; \Delta_1; \Omega \rightarrow \Xi'; \Delta'; \Gamma'$.
\end{lemma}
\begin{proof}
   Straightforward use of induction on $\Omega$ (the terms of the head) except for the sub-case of comprehensions and aggregates, where we need to use the comprehension and aggregate theorems to construct the derivation tree using $n$ applications of the corresponding construct.
\end{proof}

In the case of proving the soundness of comprehensions, we use a very identical theorem to the one used
to prove the body match soundness. However, in this case we need to reuse the continuation stack several
times (as many as many comprehensions can be applied). Using induction on the continuation stack, we get
$n$ (where $n \ge 0$) applications of the comprehension and $n \; \m{match}$ and $n \; \m{derive}$ judgments
that can be used to rebuild the derivation tree at the low level by using the $\dz \with L$, $\dz \with R$
and $\dz \lolli$ rules to fold and unfold the comprehension term. The theorem for aggregates works similarly.
