
\newcommand{\selector}[0]{[\; S \Rightarrow y; \; BE \;] \lolli HE}
\newcommand{\comprehension}[0]{\{ \; \widehat{x}; \; BE; \; SH \; \}}
\newcommand{\aggregate}[0]{[\; A \Rightarrow y; \; \widehat{x}; \; BE; \; SH_1; \; SH_2 \;]}

Table~\ref{tbl:ast} shows the abstract syntax for rules in LM.
An LM program $Prog$ consists of a set of derivation rules $\Sigma$ and a database $D$.
A derivation rule $R$ may be written as $BE \lolli HE$ where $BE$ is the body of the rule and
$HE$ is the head.
We can also explicitly universally quantify over variables in a rule using $\; \forall_{x}. R$.
If we want to control how facts are selected in the body, we may use \emph{selectors} of
the form $\selector$ (explained later).

\begin{table}[h]
\centering
\begin{tabular}{ l l c l }
  Program & $Prog$ & $::=$ & $\Sigma, D$ \\
  Set Of Rules & $\Sigma$ & $::=$ & $\cdot \; | \; \Sigma, R$\\
  Database & $D$ & $::=$ & $\Gamma; \Delta$ \\
  Rule & $R$ & $::=$ & $BE \lolli HE \; | \; \forall_{x}. R \; | \; \selector$ \\
  Body Expression & $BE$ & $::=$ & $L \; | \; P \; | \; C \; | \; BE, BE \; | \; \exists_{x}. BE \; | \; 1$\\
  Head Expression & $HE$ & $::=$ & $L \; | \; P \; | \; HE, HE \; | \; EE \; | \; CE \; | \; AE \; | \; 1$\\
  
  Linear Fact & $L$ & $::=$ & $l(\hat{x})$\\
  Persistent Fact & $P$ & $::=$ & $\bang p(\hat{x})$\\
  Constraint & $C$ & $::=$ & $c(\hat{x})$ \\
  Selector Operation & $S$ & $::=$ & $\mathtt{min} \; | \; \mathtt{max} \; | \; \mathtt{random}$\\
  
  Exists Expression & $EE$ & $::=$ & $\exists_{\widehat{x}}. SH$ \\
  Comprehension & $CE$ & $::=$ & $\comprehension$ \\
  Aggregate & $AE$ & $::=$ & $\aggregate$ \\
  Aggregate Operation & $A$ & $::=$ & $\mathtt{min} \; | \; \mathtt{max} \; | \; \mathtt{sum} \; | \; \mathtt{count}$ \\
  
  Sub-Head & $SH$ & $::=$ & $L \; | \; P \; | \; SH, SH \; | \; 1$\\
  
  Known Linear Facts & $\Delta$ & $::=$ & $\cdot \; | \; \Delta, l(\hat{t})$ \\
  Known Persistent Facts & $\Gamma$ & $::=$ & $\cdot \; | \; \Gamma, \bang p(\hat{t})$ \\
\end{tabular}
\caption{Abstract syntax of LM.}\label{tbl:ast}
\end{table}

The body of the rule, $BE$, may contain linear ($L$) and persistent ($P$) \emph{fact expressions} and
constraints ($C$). We can chain those elements by using $BE, BE$ or introduce body variables using $\exists_{x}. BE$.
Alternatively we can use an empty body by using $1$, which creates an axiom.

Fact expressions are template facts that instantiate variables
(from facts in the database) such as \texttt{visit(A)} in line 10 in Fig.~\ref{code:visit}.
Constraints are boolean expressions that must
be true in order for the rule to be fired (for example, \texttt{C~=~A~+~B}). Constraints use variables from fact expressions and are built using a small functional language that includes mathematical operations, boolean operations, external functions and literal values.

The head of a rule ($HE$) contains linear ($L$) and persistent ($P$) \emph{fact templates} which are uninstantiated facts and will derive new facts. The head can also have \emph{exist expressions} ($EE$), \emph{comprehensions} ($CE$) and \emph{aggregates} ($AE$). All those expressions
may use all the variables instantiated in the body. We can also use an empty head by choosing $1$.

\paragraph{Selectors}

When a rule body is instantiated using facts from the database, facts are picked
non-deterministically. While our system uses an implementation dependent order for
efficiency reasons, sometimes it is important to sort facts by one of the arguments
because linearity imposes commitment during rule derivation. The abstract syntax for
this expression is $\selector$, where $S$ is the selection operation and $y$ is the
variable in the body $BE$ that represents the value to be selected according to $S$.
An example using concrete syntax is as follows:

{\small
\begin{Verbatim}
[min => W | !edge(A, B), weight(A, B, W)] -o picked(A, B, W).
\end{Verbatim}
}

In this case, we order the \texttt{weight} facts by \texttt{W} in ascending order and then try
to match them. Other operations available are \texttt{max} and \texttt{random} (to force no pre-defined order).

\paragraph{Exists Expression}

Exists expressions ($EE$) are based on the linear logic term of the same name and are used to create new node addresses.
We can then use the new address to instantiate new facts for this node.  
The following example illustrates the use of the exists expression, where we derive
\texttt{perform-work} at a new node \texttt{B}.

{\small
\begin{Verbatim}
do-work(A, W) -o exists B. (perform-work(B, W)).
\end{Verbatim}
}

\paragraph{Comprehensions}

Sometimes we need to consume a linear fact and then immediately generate several facts depending on
the contents of the database. To solve this particular need, we created the concept of comprehensions, which are
sub-rules that are applied with all possible combinations of facts from the database. In a comprehension $\comprehension$, $\widehat{x}$ is a list of variables, $BE$ is the comprehension's body and $SH$ is the head.
The body $BE$ is used to generate all possible combinations for the head $SH$, according to the facts
in the database. Note that $BE$ is also locally restricted.

We have already seen an example of comprehensions in the visit program (Fig.~\ref{code:visit} line 7).
Here, we match \texttt{!edge(A, B)} using all the combinations available in the database and derive \texttt{visit(B)}
for each combination.

\paragraph{Aggregates}

Another useful feature in logic programs is the ability to reduce several facts into a single fact.
In LM we have aggregates ($AE$), a special kind of sub-rule similar to comprehensions.
In the abstract syntax $\aggregate$, $A$ is the aggregate operation, $\widehat{x}$ is the list of variables
introduced in $BE$, $SH_1$ and $SH_2$ and $y$ is the variable in the body
$BE$ that represents the values to be aggregated using $A$.
We use $\widehat{x}$ to try all the combinations of $BE$, but, in addition to deriving $SH_1$ for each combination,
we aggregate the values represented by $y$ and derive $SH_2$ only once using $y$.

Let's consider a database with the following facts and a rule:

{\small
\begin{Verbatim}
price(@1, 3). price(@1, 4). price(@1, 5).
count-prices(@1).
count-prices(A) -o [sum => P | . | price(A, P) | 1 | total(A, P)].
\end{Verbatim}
}

By applying the rule, we consume \texttt{count-prices(@1)} and
derive the aggregate which consumes all the \texttt{price(@1, P)} facts.
These are added and \texttt{total(@1,~12)} is derived.
LM provides aggregate operations such as minimum, maximum, sum, and count.
