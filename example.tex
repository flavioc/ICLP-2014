
Linear Meld (LM) is a \emph{forward chaining} logic programming language in the style of Datalog~\cite{Ullman:1990:PDK:533142}. The program is defined as a \emph{database of facts} and a set of \emph{derivation rules}.
Initially, we populate the database with the program's axioms and then determine which derivation rules can be applied by using the current database. Once a rule is applied, we derive new facts, which are then added to the database.
If a rule uses linear facts, they are consumed and thus deleted from the database.
The program stops when we reach \emph{quiescence}, that is, when we can no longer
apply any derivation rule.

The database of facts can be seen as a graph data structure where each node or vertex contains a
fraction of the database.  Since derivation rules can only manipulate facts belonging to
a node, we are able to perform independent rule derivations.

Each fact is a predicate on a tuple of \emph{values}, where the type of the predicate prescribes the types of the arguments.
LM rules are type-checked using the predicate declarations in the header of the program. LM has a simple type system that includes types such as
\emph{node}, \emph{int}, \emph{float}, \emph{string}, \emph{bool}. Recursive types such as \emph{list X} and \emph{pair X; Y} are
also allowed.

The first argument of every predicate must be typed as a \emph{node}.
For concurrency and data partitioning purposes, derivation rules are constrained by the expressions that can be written in the body.
The body of every rule can only refer to facts in the same node (same first argument).
However, the expressions in the head may refer to other nodes, as long as those nodes are instantiated in the body of the rule.

Each rule in LM has a defined priority that is inferred from its position in the source file.
Rules at the beginning of the file have higher priority. At the node level, we consider all
the new facts that have been not consider yet to create a set of \emph{candidate rules}.
The set of candidate rules is then applied (by priority) and updated as new facts are derived.

Our first program example is shown in Fig.~\ref{code:message}. This is a message routing program
that simulates message transmission through a network of nodes.
We first declare all the predicates (lines 1-2), which represent the different facts we are going to
use.
Predicate \texttt{edge/2} is a non \texttt{linear} (persistent) predicate and \texttt{message/3} is linear. While linear facts may be retracted, persistent facts are always
true once they are derived.

The program rules are declared in lines 4-8, while the program's axioms are written in lines 10-11.
The general form of a rule is $\mathtt{A_1},...,\mathtt{A_n}$ \texttt{-o} $\mathtt{B_1},...,\mathtt{B_m}$, where $\mathtt{A_1},...,\mathtt{A_n}$ are matched against local facts and $\mathtt{B_1},...,\mathtt{B_m}$ are locally asserted or transmitted to a neighboring node.
When persistent facts are used (line 5) they must be
preceded by \texttt{!} for readability.

\begin{figure}[h!]
\scriptsize\begin{Verbatim}[numbers=left]
type edge(node, node). // define direct edge
type linear message(node, string, list node). // message format

message(A, Content, [B | L]),
!edge(A, B)
   -o message(B, Content, L). // message derived at node B

message(A, Content, []) -o 1. // message received

!edge(@1, @2). !edge(@2, @3). !edge(@3, @4). !edge(@1, @3).
message(@1, 'Hello World', [@3, @4]).
\end{Verbatim}
\caption{Message program.}
  \label{code:message}
\end{figure}
\normalsize

The first rule (lines 4-6) grabs the next node in the route list (third argument of \texttt{message/3}) and
ensures that a communication edge exists (through \texttt{edge(A, B)}).
A new \texttt{message(B,~Content,~L)} fact is derived at node \texttt{B}.
When the route list is empty, the message has reached its destination and thus it is consumed
(rule in lines 8-9). Note that the '\texttt{1}' in the head of the rule on line 8 means that nothing is derived.

Figure~\ref{code:visit} presents another complete LM program which given a graph
of nodes visits all nodes reachable from node $@1$.
The first rule of the program (lines 6-7) is fired when a node \texttt{A} has both the \texttt{visit(A)} and \texttt{unvisited(A)} facts.
When fired, we first derive \texttt{visited(A)} to mark node \texttt{A} as \textit{visited} and use a
\emph{comprehension} to go through all the edge facts \texttt{edge(A,B)} and derive \texttt{visit(B)} for each
one (comprehensions are explained next in detail). This forces those nodes to be visited.
The second rule (line 9) is fired when a
node \texttt{A} is already visited more than once: we keep the \texttt{visited(A)} fact and delete \texttt{visit(A)}.
Line 13 starts the process by asserting the \texttt{visit(@1)} fact.

\begin{figure}[h!]
\scriptsize\begin{Verbatim}[numbers=left]
type edge(node, node).
type linear visit(node).
type linear unvisited(node).
type linear visited(node).

visit(A), unvisited(A) // mark node as visited and visit neighbors
   -o visited(A), {B | !edge(A, B) | visit(B)}.

visit(A), visited(A) -o visited(A). // already visited

!edge(@1, @2). !edge(@2, @3). !edge(@1, @4). !edge(@2, @4).
unvisited(@1). unvisited(@2). unvisited(@3). unvisited(@4).
visit(@1).
\end{Verbatim}
  \caption{Visit program.}
  \label{code:visit}
\end{figure}
\normalsize

If the graph is connected, it is easy to prove that every node \texttt{A} will derive \texttt{visited(A)},
regardless of the order in which rules are applied.

\vspace{-4mm}