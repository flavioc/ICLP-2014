
We now present some LM programs in order to show that they tend to
be concise and easy to write\footnote{More examples of LM programs are available at \url{http://github.com/flavioc/meld}.}. We want to make clear how the
language facilities can be used to solve interesting graph-based algorithms.

\begin{comment}
\subsection{Bipartiteness Checking}

The problem of checking if a graph is bipartite can be seen as a 2-color graph coloring problem.
The code for this algorithm is shown in Appendix~\ref{code:bichecking}. All nodes in the graph
start as \texttt{unchecked}, because they do not have a color yet. The axiom \texttt{visit(@1, 1)} is
instantiated at node \texttt{@1} (line 9) in order to color this node with color 1.

If a node is \texttt{unchecked} and needs to be marked with a color \texttt{P} then the rule in
lines 11-12 is applied. We consume the \texttt{unchecked} fact and derive a \texttt{checked(A, P)}
to effectively color the node with \texttt{P}. We also derive \texttt{visit(B, next(P))} in
our neighbor nodes in order to color them with the other color.

The coloring can fail if a node is already colored with a color \texttt{P} and needs to be colored
with a different color (line 15) or if it has already failed (line 16).
\end{comment}

\subsection{Shortest Distance}

Finding the shortest distance between two nodes in a graph is another well known graph problem.
Appendix~\ref{code:shortest_path} presents the LM code for this algorithm. We use an \texttt{edge}
predicate to represent directed edges between nodes and their corresponding weights. To represent the shortest
distance to a node \texttt{startnode} we have a \texttt{path(A, B, F)} where \texttt{B} is the distance to \texttt{startnode}
and \texttt{F} is a flag to indicate if such distance has been propagated to the neighbors. Since the distance from
the \texttt{startnode} to itself is \texttt{0}, we start with the axiom \texttt{path(startnode, 0, notused)} to kick-start
the algorithm.

The first rule avoids propagating paths with the same distance and the second rule eliminates paths where the distance
is already larger than some other distance. Finally, the third rule, marks the path as \texttt{used} and propagates
the distance to the neighboring nodes by taking into account the edge weights.

With enough time, the program will reach quiescence and the shortest distance between \texttt{startnode} and \texttt{finalnode}
will be determined.

\subsection{PageRank}

PageRank is a well known graph algorithm that is used to compute the relative relevance of web pages.
The code for a synchronous version of the algorithm is shown in Appendix~\ref{code:pagerank}.
As the name indicates, the pagerank is computed for a certain number of iterations. Initially, the initial pagerank is the same for every page and is
initialized in the first rule (lines 15-18) along with an accumulator.

The second rule of the program (lines 21-25) propagates a newly computed pagerank value to all neighbors. Consequently, each node will accumulate
the pagerank values that are sent to them through the fourth rule (lines 35-37) and it will immediately add all the other available incoming values
through the use of the aggregate. When we have accumulated all the values we need, the third rule (lines 28-32) is fired and a new pagerank value is derived.

We also have an asynchronous version of the algorithm that trades correctness with convergence speed since it does synchronize between iterations.

\subsection{N Queens}

The N queens~\cite{8queens} puzzle is the problem of placing N chess queens on an NxN chessboard so
that no pair of two queens attack each other. The specific challenge of finding all the distinct
solutions to this problem is a good benchmark in designing parallel algorithms. The LM solution is presented
in Appendix~\ref{code:nqueens}.

First, we consider each cell of the chessboard as a node that can communicate with the adjacent left
(\texttt{left}) and adjacent right (\texttt{right}) cells and also with the first two non-diagonal cells in the next row
(\texttt{down-left} and \texttt{down-right}). For instance, the node at cell \texttt{(0,~3)} (fourth cell in the first row) will connect
to cells \texttt{(0,~2)}, \texttt{(0,~4)} and also \texttt{(1,~1)} and \texttt{(1,~5)}, respectively. The states are represented as a list
of integers, where each integer is the column number where the queen was placed. For example \texttt{[2, 0]}
means that a queen is placed in cell \texttt{(0,~0)} and another in cell \texttt{(1,~2)}.

An empty state is instantiated in the top-left node \texttt{(0,~0)} and then propagated to all nodes in the same row (lines 20-22).
Each node then tries to place a queen on their cell and then send a new state to the row below (lines 63-65).
Recursively, when a node receives a new state, it will (i) send the state to the left
or to the right and (ii) try to place the queen in its cell (using \texttt{test-y}, \texttt{test-diag-left} and \texttt{test-diag-right}). With this method,
all states will be computed since we have facts for each valid state
at that point. When a cell cannot place a queen, that state is deleted (lines 31, 42 and 54).
When the program ends, the states will be placed in the bottom row (lines 59-61).

We find our solution very elegant, since it can be easily distributed or executed in parallel.
To the best of our knowledge it is also an unusual approach to this problem.

\begin{comment}
\subsection{Quick-Sort}

The quick-sort algorithm is a divide and conquer sorting algorithm that works by splitting
a list of items into two sublists and then recursively sorting the two sublists.
To split the list, we pick a pivot element and put the items that are smaller than the pivot
into the first sublist and items greater than the pivot into the second list.

The quick-sort algorithm is interesting because it does not map immediately to the graph-based
model of LM. Our approach considers that the program starts with a single node where
the initial list is located. Then we split the list as usual and create two nodes
that will recursively sort the sublists. Interestingly, this will create a tree
that will look similar to a call tree in a functional language.

Appendix~\ref{code:quicksort} presents the code for the quick-sort algorithm in LM.
For each sublist to sort, we start with a \texttt{down} fact that must be (eventually)
transformed into an \texttt{up} fact, where the sublist in the \texttt{up} fact is sorted.
In line 11 we start with the initial list at node \texttt{@0}. Lines 13-16 will immediately
sort the list when the number of items is very small. Otherwise, we apply the rule in line 17.
\texttt{buildpivot} will first split the list using the pivot \texttt{X} using rules in
lines 23-26. When there is nothing more to split, we apply the rule in lines 19-21
that uses an exists constructor to create nodes \texttt{B} and \texttt{C}. The sublists
are then sent to these nodes using \texttt{down} facts. Note, however, that we also
derive \texttt{back} facts, that will be used to send the sorted list back using the rule
in line 40.

When the sublists are finally sorted, we get two \texttt{sorted} facts that will match
against \texttt{waitpivot} in the rule located in lines 28-31. The sorted sublists
are appended and then an \texttt{up} fact is finally derived (line 37).

\end{comment}