
We now present LM programs in order to illustrate common programming techniques\footnote{More examples of LM programs are available at \url{http://github.com/flavioc/meld}.}.

\paragraph{Shortest Distance}

Finding the shortest distance between two nodes in a graph is another well known graph problem.
Fig.~\ref{code:shortest_path} presents the LM code to solve this particular problem.

We use an \texttt{edge/3}
predicate to represent directed edges between nodes and their corresponding weights. To represent the shortest
distance to a node \texttt{startnode} we have a \texttt{path(A,D,F)} where \texttt{D} is the distance to \texttt{startnode}
and \texttt{F} is a flag to indicate if such distance has been propagated to the neighbors. Since the distance from
the \texttt{startnode} to itself is \texttt{0}, we start the algorithm with the axiom \texttt{path(startnode,0,notused)}.

The first rule avoids propagating paths with the same distance and the second rule eliminates paths where the distance
is already larger than some other distance. Finally, the third rule, marks the path as \texttt{used} and propagates
the distance to the neighboring nodes by taking into account the edge weights.
Eventually, the program will reach quiescence and the shortest distance between \texttt{startnode} and \texttt{finalnode}
will be determined.

\newcommand{\BigO}[1]{\ensuremath{\operatorname{O}\bigl(#1\bigr)}}

\begin{figure}[h]
\scriptsize\begin{Verbatim}[numbers=left]
type edge(node, node, int).
type linear path(node, int, int).

const used = 1.
const notused = 0.

path(startnode, 0, notused).

path(A, D, used), path(A, D, notused)
   -o path(A, D, used).

path(A, D1, X), path(A, D2, Y), D1 <= D2
   -o path(A, D1, X). // keep the shorter distance

path(A, D, notused), A <> finalnode
   -o {B, W | !edge(A, B, W) | path(B, D + W, notused)}, path(A, D, used). // propagate new distance
\end{Verbatim}
\caption{Shortest Distance Program.}
\label{code:shortest_path}
\end{figure}
\normalsize

In the worst case, this algorithm runs in \BigO{N E}, where $N$ is the number of nodes and $E$ is the
number of edges. If we decide to always propagate the shortest distance of the graph, we get Dijkstra's algorithm~\cite{Dijkstra}. However, this is
not feasible, since we would need to globally decide which node to run next, removing concurrency.

\paragraph{PageRank}

PageRank~\cite{Page:2001:MNR} is a well known graph algorithm that is used to compute the relative relevance of web pages.
The code for a synchronous version of the algorithm is shown in Fig.~\ref{code:pagerank}.
As the name indicates, the pagerank is computed for a certain number of iterations. The initial pagerank is the same for every page and is
initialized in the first rule (line 12) along with an accumulator.


\begin{figure}[h]
   \scriptsize\begin{Verbatim}[numbers=left]
type output(node, node, float).
type linear pagerank(node, float, int).
type numLinks(node, int).
type numInput(node, int).
type linear accumulator(node, float, int, int).
type linear newrank(node, node, float, int).
type linear start(node).

start(A).

start(A), !numInput(A, T)
   -o accumulator(A, 0.0, T, 1), pagerank(A, 1.0 / float(@world), 0).

pagerank(A, V, Id), !numLinks(A, C), Id < iterations, Result = V / float(C)
   -o {B, W | !output(A, B, W) | newrank(B, A, Result, Id + 1)}. // propagate new pagerank value

accumulator(A, Acc, 0, Id), !numInput(A, T), V = 0.85 + 0.15 * Acc, Id <= iterations
   -o pagerank(A, V, Id), accumulator(A, 0.0, T, Id + 1). // new pagerank value
	
newrank(A, B, V, Id), accumulator(A, Acc, T, Id), T > 0
   -o [sum => S, count => C | D | newrank(A, D, S, Id) | 1 | accumulator(A, Acc + V + S, T - 1 - C, Id)].
\end{Verbatim}
\caption{Synchronous PageRank program.}
\label{code:pagerank}
\normalsize
\end{figure}

The second rule of the program (lines 14-15) propagates a newly computed pagerank value to all neighbors. Each node will then accumulate
the pagerank values that are sent to them through the fourth rule (lines 20-21) and it will immediately add other currently available values
through the use of the aggregate. When we have accumulated all the values we need, the third rule (lines 17-18) is fired and a new pagerank value is derived.

%We also have an asynchronous version of the algorithm that trades correctness with convergence speed since it does not synchronize between iterations.

\paragraph{N-Queens}

The N-Queens~\cite{8queens} puzzle is the problem of placing N chess queens on an NxN chessboard so
that no pair of two queens attack each other. The specific challenge of finding all the distinct
solutions to this problem is a good benchmark in designing parallel algorithms. The LM solution is presented
in \ref{code:nqueens}.

First, we consider each cell of the chessboard as a node that can communicate with the adjacent left
(\texttt{left}) and adjacent right (\texttt{right}) cells and also with the first two non-diagonal cells in the next row
(\texttt{down-left} and \texttt{down-right}). For instance, the node at cell \texttt{(0,~3)} (fourth cell in the first row) will connect
to cells \texttt{(0,~2)}, \texttt{(0,~4)} and also \texttt{(1,~1)} and \texttt{(1,~5)}, respectively. The states are represented as a list
of integers, where each integer is the column number where the queen was placed. For example \texttt{[2, 0]}
means that a queen is placed in cell \texttt{(0,~0)} and another in cell \texttt{(1,~2)}.

An empty state is instantiated in the top-left node \texttt{(0,~0)} and then propagated to all nodes in the same row (lines 19-20).
Each node then tries to place a queen on their cell and then send a new state to the row below (lines 52-54).
Recursively, when a node receives a new state, it will (i) send the state to the left
or to the right and (ii) try to place the queen in its cell (using \texttt{test-y}, \texttt{test-diag-left} and \texttt{test-diag-right}). When a cell cannot place a queen, that state is deleted (lines 29, 37 and 45).
When the program ends, the states will be placed in the bottom row (lines 49-50).

Most parallel implementations distribute the search space of the problem by assigning incomplete boards as tasks to workers.
Our approach is unusual because our tasks are the cells of the board.